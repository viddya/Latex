\documentclass{article}
\begin{document}
A quadtree is a tree data structure in which every node has  four children. Quadtrees are most often used to partition a two-dimensional space by recursively subdividing it into four quadrants or regions. 
Each node corresponds to a square in the plane and each child of a node corresponds to a quadrant(Northwest,Northeast,Southwest or Southeast) of the square of the node.

Quadtrees can be used to store different types of data such as points, lines, areas , surfaces or rasters.
This paper invesitigates point quadtrees.

Recursive Construction of Point Quadtrees:

The Quadtree built can either be a balanced or an unbalanced quadtree depending on the distribution of data points.
This method sequentially processes each point from a set of points.
The maximum level of the Quadtree is predetermined.
The root node contains all the points .It is at level 0 / top of the Quadtree and it does not have a parent node.The leaf node is at the maximum level of the Quadtree and it does not have any links.
The link nodes are the nodes which have a parent and child nodes(these nodes will have a maximum of four children).

The direction /quadrant (Northwest - NW,Northeast-NE,Southwest-SW or Southeast-SE) of the point is determined and the node is built based on the direction.

Move to that node which is at the level 1 and find the direction of the point within this square of the node, and repeat it until the maximum level of the Quadtree is reached

Repeat the above steps till all the points in the point set is processed.

Each leaf node contains a list of point buffers. The point buffer is assigned a maximum number of points that it can hold.If a buffer is full, then the points are moved to the next buffer in the list.

The root node is subdivided recursively till the leaf node is reached. The Quadtree with k levels including the root would have 4(k?1) nodes on the kth level and (4k) ? 1 nodes in total.
In this project, no of levels,  k = 4.

The side lengths of squares in a Quadtree halve with increasing depth. If 'S' is the length of a side of the root node, then at depth 'd', the length of a side of the square at the level is S/2^d.

























\end{document}