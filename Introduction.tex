Introduction:

Range search algorithms, which make use of spatial data structures, perform much better than the ones that do not partition the data before processing. Quadtree is a hierarchical spatial data structure that is used for both indexing and compressing geographic database layers due its applicability to many types of data,its ease of implementation and relatively good performance. The point quadtree structure recursively divides space into equal four rectangles based on the location of the points. The root of the quadtree represents the whole 2D space. Successive points divide each new sub-region into quadrants until all points are indexed.

How quadtree reduces complexity O(N)?
As done traditionally, the quadtree is built on the CPU. In order to increase the threshold on the number of region of interest/polygons processed within a given time frame, we need to parallelize the traversal of the quadtree. Using a sequential approach here will demand an increasing processor speed, which in turn will lead to increased power consumption. This increased demand for power cannot be met due to the limitations in the properties of the silicon material. Therefore parallelism is the only way to expand the number of regions/polygons processed within a given time frame. 
We use NVIDIA?s CUDA programming model for the GPU, which allows us to execute thousands of threads simultaneously on the GPU. In CUDA, the CPU executes the sequential part of the code and the computationally intensive part, which has more scope for parallelization, is done in the GPU. The CPUs have sophisticated control logic and large cache memories and are optimized for a sequential approach. GPUs are designed for a high throughput as they have lesser control logic implementation per thread and smaller caches. And GPUs have much higher bandwidth compared to CPU. Therefore GPUs would the better choice to implement the range search using quadtree for a large number of points.
